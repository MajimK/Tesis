\documentclass{article}
\usepackage{amsmath}
\usepackage{graphicx}
\usepackage{multicol}
\usepackage{amsmath}
\usepackage{tcolorbox}
\usepackage{subcaption}
\usepackage[spanish]{babel}
\usepackage{lipsum}


\begin{document}
\section{Preliminares}
\subsection{Problema de Enrutamiento de Vehiculos (VRP)}
El Problema de Enrutamiento de Vehiculos (VRP) es uno de los problemas fundamentales en la optimización combinatoria, ampliamente estudiado por su importancia tanto práctica como teórica. Formulado por Dantzig y Ramser en 1959, el VRP consiste en diseñar rutas óptimas para una flota de vehículos que deben satisfacer la demanda de un conjunto de clientes dispersos geográficamente.\cite{ref2} Estos vehículos parten y regresan a uno o varios depósitos, con el objetivo de minimizar un criterio de costo, como la distancia total recorrida, el tiempo de operación o el consumo de recursos \cite{ref3}. El VRP tiene aplicaciones prácticas en áreas como la logística, el transporte urbano y la distribución de bienes en grandes cadenas de suministro.\cite{ref8,ref10,ref11}

Una característica esencial del VRP es su clasificación como problema NP-duro, lo que significa que encontrar una solución óptima es computacionalmente intractable para instancias grandes. Por ello, en la práctica, se recurre a heurísticas y metaheurísticas para obtener soluciones aproximadas con alta calidad en tiempos razonables.\cite{ref7}

\subsubsection{Variantes del VRP}

Dado que las aplicaciones del VRP en el mundo real presentan restricciones y características específicas, han surgido múltiples variantes del problema. Entre las más estudiadas se encuentran:
\begin{enumerate}
\item
\textbf{Problema de Enrutamiento de Vehículos Capacitado (CVRP):}\\
En esta variante, cada vehículo tiene una capacidad máxima que no puede ser excedida por la suma de las demandas de los clientes asignados a su ruta. Introducida como una extensión directa del VRP, el CVRP es relevante en contextos logísticos donde el volumen o peso de los productos transportados juega un rol crucial, como en la distribución de alimentos o materiales de construcción. El principal desafío en el CVRP es balancear el uso eficiente de los vehículos mientras se minimizan los costos totales, respetando las restricciones de capacidad.\cite{ref6}

\item
\textbf{Problema de Enrutamiento de Vehículos con Ventanas de Tiempo (VRPTW):}\\
Aquí, además de optimizar las rutas, se deben cumplir ventanas de tiempo específicas durante las cuales cada cliente puede ser atendido. Esta variante es común en servicios de entrega, como el correo o las empresas de paquetería, donde los clientes especifican franjas horarias para recibir sus pedidos. El VRPTW introduce complejidad adicional al problema, ya que las soluciones deben sincronizar las rutas y los tiempos de servicio, minimizando penalizaciones por retrasos o tiempos de espera.\cite{ref9}

\item
\textbf{Problema de Enrutamiento de Vehículos con Múltiples Depósitos (MDVRP):}\\
En el MDVRP, los vehículos pueden partir de múltiples depósitos en lugar de un único punto central. Esto refleja escenarios reales donde empresas con centros de distribución regionales buscan optimizar sus operaciones de manera integrada. La asignación de clientes a depósitos y la planificación de rutas eficaces son desafíos clave en esta variante.\cite{ref1}

\item
\textbf{Problema de Enrutamiento de Vehículos Abierto (OVRP):}\\
En esta variante, los vehículos no están obligados a regresar al depósito al finalizar sus rutas. El OVRP modela situaciones donde los vehículos pueden terminar sus recorridos en ubicaciones remotas, como entregas a largo alcance en transporte intermodal. Esto reduce costos al evitar viajes innecesarios de retorno, pero complica la logística, ya que los puntos finales deben estar estratégicamente definidos.

\item
\textbf{Problema de Enrutamiento de Vehículos con Entregas Divididas (SDVRP):}
A diferencia del VRP clásico, en el SDVRP se permite que un cliente sea atendido por múltiples vehículos. Esta variante es útil cuando las demandas de ciertos clientes superan la capacidad de un solo vehículo o cuando se busca maximizar la flexibilidad operativa. Sin embargo, el SDVRP introduce complejidad adicional en la planificación, al requerir una coordinación eficiente entre los vehículos.
Métodos de Solución
\end{enumerate}

Dada la complejidad del VRP y sus variantes, se han desarrollado múltiples enfoques para resolverlos. Los métodos exactos, como la programación lineal entera y el branch-and-bound, son efectivos para instancias pequeñas, pero su escalabilidad es limitada \cite{ref5}. En instancias más grandes, se emplean heurísticas y metaheurísticas como la búsqueda de vecindad variable (VNS), la búsqueda en vecindades amplias (LNS)\cite{ref4}, los algoritmos genéticos y la colonia de hormigas.\cite{ref7} Estas técnicas buscan un balance entre la exploración del espacio de soluciones y la explotación de las áreas más prometedoras.

Los recientes avances incluyen el uso de métodos híbridos y técnicas basadas en aprendizaje automático para mejorar la eficiencia y robustez de los algoritmos. Además, la creación de conjuntos de datos de prueba más complejos y diversos, como los benchmarks de \cite{ref10}, ha permitido evaluar y comparar diferentes enfoques de manera más rigurosa, impulsando el progreso en el campo.

\subsection{Criterios de vecindad}

En el contexto del Vehicle Routing Problem (VRP) y sus variantes, los criterios de vecindad juegan un papel fundamental dentro de los algoritmos de búsqueda local y metaheurísticas. Estos criterios definen cómo se generan y exploran las soluciones vecinas a una solución actual, permitiendo mejorarla gradualmente o escapar de óptimos locales. En esencia, un criterio de vecindad establece las reglas mediante las cuales se modifica una solución, y la elección de criterios adecuados es crucial para la eficacia de los métodos de optimización.
Ejemplos de Criterios de Vecindad
\begin{enumerate}
\item
    \textbf{Reubicación de un cliente (Relocation):}
    Este criterio consiste en mover un cliente de una ruta a otra o a una posición diferente dentro de la misma ruta. Por ejemplo, un cliente puede ser reubicado para equilibrar las cargas de los vehículos o para reducir el costo de la ruta.
\item
    \textbf{Intercambio de clientes (Swap):}
    En este caso, se intercambian dos clientes entre diferentes rutas o dentro de la misma ruta. Esto permite ajustar las rutas para optimizar la distancia total recorrida o cumplir restricciones específicas, como las de capacidad o tiempo.
\item
    \textbf{Intercambio de secuencias (2-opt y 3-opt):}
    Estos criterios implican modificar subrutas intercambiando dos o tres aristas, respectivamente. Por ejemplo, en un movimiento 2-opt, se eliminan dos aristas de la solución actual y se reconectan los nodos para formar una ruta diferente. Este tipo de criterio es común en problemas como el TSP y se adapta bien al VRP para minimizar la distancia total.
\item
    \textbf{Inserción de clientes (Insertion):}
    Este criterio selecciona un cliente que no está incluido en la solución actual o que ha sido eliminado temporalmente, y lo inserta en una posición que minimice el costo de la ruta. Este enfoque es particularmente útil en heurísticas constructivas y métodos como la Búsqueda en Vecindades Amplias (LNS).
\item
    \textbf{Intercambio múltiple (Generalized Swap):}
    Extiende el concepto de intercambio a grupos de clientes. Por ejemplo, se pueden intercambiar secuencias completas de clientes entre rutas para explorar vecindarios más amplios.
\end{enumerate}

\subsubsection{Importancia de los Criterios de Vecindad}

Los criterios de vecindad son esenciales para equilibrar la intensificación y la diversificación en la búsqueda. La intensificación se enfoca en explorar soluciones cercanas para mejorar la calidad local, mientras que la diversificación permite moverse a regiones distantes del espacio de soluciones para evitar quedarse atrapado en óptimos locales. Técnicas como la Búsqueda de Vecindad Variable (VNS) aprovechan conjuntos de criterios de vecindad para cambiar sistemáticamente entre diferentes vecindarios, mejorando así la eficacia de la búsqueda.

En el VRP, los criterios de vecindad también permiten incorporar restricciones específicas del problema, como ventanas de tiempo o capacidades de los vehículos, ajustando las operaciones de modificación para asegurar la factibilidad de las soluciones. Esto hace que sean una herramienta versátil y fundamental en la optimización combinatoria.

\subsection{Busqueda en Vecindades Variables (VNS)}

La Busqueda en Vecindades Variables (VNS) es una metaheurística de optimización introducida por Mladenović y Hansen en 1997, diseñada para resolver problemas de optimización combinatoria y global mediante la exploración sistemática de múltiples vecindarios en el espacio de soluciones. A diferencia de otros métodos de búsqueda local, que suelen trabajar con un único vecindario, VNS se basa en la idea de que cambiar dinámicamente entre diferentes estructuras de vecindad permite escapar de óptimos locales y explorar de manera más efectiva el espacio de soluciones. Este enfoque ha demostrado ser particularmente efectivo en problemas NP-duros, como el Vehicle Routing Problem (VRP) y sus variantes.

El método VNS opera mediante una secuencia iterativa de tres fases principales. Primero, se selecciona una solución inicial, que puede generarse de forma aleatoria o utilizando una heurística constructiva. A continuación, en la fase de perturbación, se aplica un operador que modifica la solución actual, cambiando su vecindario de acuerdo con una estructura predeterminada. En la siguiente etapa, se utiliza una búsqueda local para explorar el vecindario perturbado en busca de una solución mejor. Finalmente, si la nueva solución es superior, esta se convierte en la solución actual, y el proceso se repite; en caso contrario, el algoritmo cambia a un nuevo vecindario. Este cambio sistemático permite explorar de manera exhaustiva y estratégica el espacio de soluciones.

En el contexto del VRP y sus variantes, VNS ha sido ampliamente utilizado para encontrar soluciones cercanas al óptimo. Por ejemplo, en el Capacitated Vehicle Routing Problem (CVRP), las perturbaciones suelen implicar el intercambio de clientes entre rutas o la reasignación de clientes dentro de una misma ruta. En el VRP con ventanas de tiempo (VRPTW), VNS se adapta para cumplir con restricciones temporales, incorporando operadores que ajusten las secuencias de visitas según las ventanas de tiempo permitidas. Adicionalmente, en variantes como el \textit{Multi-Depot VRP} (MDVRP), los vecindarios incluyen movimientos que redistribuyen clientes entre depósitos, optimizando las rutas de manera global.

Una de las principales ventajas de VNS es su simplicidad y flexibilidad, ya que puede integrarse fácilmente con otros enfoques de optimización. Sin embargo, su efectividad depende de un diseño adecuado de los vecindarios y de los operadores utilizados, ya que estos determinan la calidad de las soluciones exploradas y la capacidad del algoritmo para escapar de óptimos locales. Recientes investigaciones han introducido variantes avanzadas como la Búsqueda de Vecindad Infinitamente Variable (IVNS), que permite explorar vecindarios dinámicamente generados mediante un lenguaje formal, ampliando aún más el alcance de esta técnica.

VNS sigue siendo una herramienta valiosa en la optimización combinatoria, particularmente en problemas donde la calidad de la solución y el tiempo de ejecución son críticos. Su enfoque sistemático para explorar el espacio de soluciones lo convierte en un método robusto y eficiente, adecuado para una amplia gama de aplicaciones prácticas e industriales.

\subsection{Busqueda en Vecindades Grandes (LNS)}
La Búsqueda en Vecindarios Amplios (Large Neighborhood Search, LNS) es una metaheurística diseñada para resolver problemas de optimización combinatoria, como el Problema de Enrutamiento de Vehículos (VRP), permitiendo la exploración de grandes porciones del espacio de soluciones. Introducida por Shaw en 1998, LNS se basa en un enfoque innovador que combina la destrucción y reconstrucción de soluciones para escapar de óptimos locales, un desafío recurrente en problemas complejos de optimización.

El principio central de LNS es dividir el proceso de búsqueda en dos etapas complementarias. En la primera etapa, denominada destrucción, se elimina una parte sustancial de la solución actual, generando una solución parcial que abre la posibilidad de explorar nuevas configuraciones. Posteriormente, en la etapa de reconstrucción, la solución incompleta se completa nuevamente mediante el uso de heurísticas específicas o técnicas optimizadoras que intentan mejorar el resultado inicial. Este ciclo de destrucción y reconstrucción es clave para explorar un vecindario amplio del espacio de búsqueda, lo que diferencia a LNS de métodos tradicionales que solo realizan cambios pequeños y locales.

En el contexto del VRP, LNS ha sido ampliamente aplicado debido a su capacidad para manejar instancias grandes y problemas con múltiples restricciones. Por ejemplo, en el CVRP, el método se utiliza para ajustar las rutas de los vehículos mientras se respeta la capacidad máxima de carga. De manera similar, para variantes como el VRP con ventanas de tiempo (VRPTW), LNS reconfigura los itinerarios para garantizar que las entregas cumplan con los intervalos de tiempo especificados por los clientes. Además, en situaciones dinámicas donde los pedidos pueden surgir en tiempo real, LNS permite reorganizar las rutas adaptándose rápidamente a los cambios.

Entre las principales ventajas de LNS destaca su flexibilidad, ya que se puede personalizar fácilmente para abordar variantes específicas del problema, adaptando los mecanismos de destrucción y reconstrucción según las características del escenario. Además, su capacidad para diversificar el espacio de búsqueda lo hace especialmente útil para evitar el estancamiento en óptimos locales, un problema común en métodos de búsqueda más restrictivos. Sin embargo, también presenta limitaciones. La exploración de grandes vecindarios puede ser computacionalmente intensiva, especialmente en problemas de gran escala, lo que puede requerir una cantidad significativa de recursos para alcanzar soluciones de calidad. Asimismo, la eficacia de LNS depende en gran medida de las heurísticas utilizadas durante la etapa de reconstrucción, ya que estas determinan la calidad de las nuevas soluciones generadas.\cite{ref4}

La implementación de LNS sigue un flujo bien definido. El proceso inicia con la creación de una solución inicial factible que sirve como punto de partida. A continuación, se alterna entre la eliminación de una parte de la solución y la reconstrucción de los elementos eliminados. En cada iteración, se analiza si la nueva solución debe sustituir a la actual, basándose en criterios como el costo total de las rutas o la viabilidad del plan. Este ciclo continúa hasta que se cumple un límite previamente establecido, ya sea en tiempo o en número de iteraciones. Este enfoque equilibra la exploración de nuevas áreas del espacio de soluciones con la explotación de las soluciones prometedoras ya descubiertas.

En términos de impacto, LNS ha demostrado ser una de las herramientas más eficaces para abordar problemas complejos de ruteo, especialmente aquellos donde las restricciones operativas y las necesidades de optimización dificultan la aplicación de métodos exactos. Su capacidad para combinar flexibilidad, escalabilidad y adaptabilidad lo ha convertido en una técnica preferida tanto en la investigación académica como en aplicaciones prácticas, como la logística, la distribución y la gestión del transporte.

\subsection{GLPK}
El GNU Linear Programming Kit (GLPK) es una biblioteca de software diseñada para resolver problemas de programación lineal (PL) y programación lineal entera mixta (MIP). GLPK es parte del proyecto GNU y se distribuye bajo la Licencia Pública General de GNU, lo que lo convierte en una opción de código abierto ampliamente adoptada tanto en entornos académicos como industriales. Ofrece herramientas robustas y flexibles para modelar y resolver problemas de optimización combinatoria, siendo particularmente útil en aplicaciones como logística, transporte y gestión de recursos.

GLPK se centra en la resolución de problemas formulados en términos de modelos matemáticos, donde una función objetivo debe ser maximizada o minimizada bajo un conjunto de restricciones lineales. Estos modelos se especifican típicamente utilizando el lenguaje de modelado GNU Math Programming Language (GMPL), similar en sintaxis a lenguajes como AMPL. GMPL permite describir problemas de manera declarativa y concisa, separando la definición del modelo de los datos específicos, lo que facilita la reutilización y la adaptación del modelo a diferentes escenarios.

Una de las características destacadas de GLPK es su implementación del método simplex revisado y del método primal-dual para resolver problemas de programación lineal. Para problemas más complejos, como los de programación lineal entera mixta, GLPK utiliza un enfoque de ramificación y corte (branch-and-cut). Este enfoque combina técnicas de ramificación para dividir el problema en subproblemas más pequeños y métodos de corte para eliminar regiones inviables del espacio de búsqueda, mejorando la eficiencia del proceso.

En el contexto del Problema de Enrutamiento de Vehículos (VRP) y sus variantes, GLPK puede ser utilizado para modelar y resolver formulaciones exactas. Por ejemplo, el VRP Capacitado (CVRP) se puede representar como un problema de programación lineal entera mixta, donde las restricciones garantizan que cada cliente sea atendido por un único vehículo y que las capacidades no se excedan. GLPK permite implementar estas restricciones de manera directa y explorar soluciones óptimas para instancias pequeñas o medianas. Además, cuando se combina con metaheurísticas como la Búsqueda de Vecindad Variable (VNS) o la Búsqueda en Vecindarios Amplios (LNS), GLPK puede proporcionar soluciones iniciales de alta calidad o refinar las soluciones generadas.

Entre las principales ventajas de GLPK se encuentra su flexibilidad para trabajar con diversos tipos de problemas de optimización y su integración con otros lenguajes de programación como Python y C++, a través de interfaces y bindings. Esto lo convierte en una herramienta poderosa para investigadores y desarrolladores que necesitan soluciones personalizadas y escalables. Sin embargo, su rendimiento puede ser limitado en instancias de gran escala o problemas extremadamente complejos debido a la naturaleza computacionalmente intensiva de los métodos exactos que utiliza.

GLPK ha demostrado ser una herramienta esencial en el campo de la optimización, proporcionando una plataforma accesible y confiable para abordar problemas desafiantes. Su uso en combinación con otras metodologías, como metaheurísticas y técnicas de descomposición, amplía su aplicabilidad en la investigación operativa y la industria.

\subsection{Lisp}
LISP (List Processing) es uno de los lenguajes de programación más antiguos, creado por John McCarthy en 1958 para el desarrollo de aplicaciones relacionadas con la inteligencia artificial (IA). Su flexibilidad, simplicidad y capacidad para manejar estructuras simbólicas lo han consolidado como una herramienta poderosa para abordar problemas complejos, incluidos los relacionados con la optimización y el modelado en investigación operativa, como el Problema de Enrutamiento de Vehículos (VRP) y sus variantes, como el CVRP.

LISP se distingue por su enfoque en el procesamiento de listas y su propiedad de homoiconicidad, que le permite tratar el código como datos. Esto posibilita que los programas en LISP puedan modificarse a sí mismos durante su ejecución, una característica clave para escenarios donde los algoritmos requieren adaptarse dinámicamente o explorar espacios de búsqueda complejos. Además, su compatibilidad con la programación funcional facilita la creación de algoritmos concisos y reutilizables.

En el contexto del VRP, LISP se ha empleado en combinación con métodos exactos y heurísticos para modelar y resolver problemas. Su habilidad para trabajar con estructuras simbólicas y listas lo hace ideal para implementar algoritmos de búsqueda como la Búsqueda de Vecindad Variable (VNS) y la Búsqueda en Vecindarios Amplios (LNS). Asimismo, en problemas que demandan la exploración de grandes vecindarios o la gestión de restricciones complejas, LISP permite desarrollar funciones que manipulen soluciones de manera eficiente y elegante.

Un ejemplo destacado es el uso de LISP para crear frameworks personalizados para el CVRP. Su sistema de macros, que permite extender el lenguaje dinámicamente, lo convierte en una opción ideal para implementar operaciones como la modificación de rutas, el intercambio de clientes y la evaluación de costos en iteraciones rápidas. Por ejemplo, es posible definir estructuras simbólicas para representar rutas y usar funciones recursivas para evaluar el impacto de los cambios en tiempo real.

Otra ventaja significativa de LISP en problemas de optimización es su capacidad para integrar librerías externas como GLPK, lo que permite combinar sus funcionalidades con métodos de programación lineal o entera. Esto facilita el trabajo con problemas híbridos, donde las soluciones iniciales generadas mediante heurísticas en LISP se refinan usando métodos exactos. Además, implementaciones como SBCL (Steel Bank Common Lisp) ofrecen un alto rendimiento, lo que hace que LISP sea competitivo en términos de velocidad para aplicaciones prácticas.

Aunque lenguajes más modernos como Python o Julia han ganado popularidad en ciertos campos, LISP sigue siendo una opción potente para proyectos que requieren adaptabilidad y eficiencia en la manipulación de estructuras complejas. Su papel en la investigación operativa y su aplicación a problemas como el VRP resaltan su relevancia continua en la resolución de problemas avanzados de optimización.


\bibliographystyle{plain}
\bibliography{bibliography} 
\end{document}




