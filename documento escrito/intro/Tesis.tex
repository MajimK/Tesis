\documentclass{article}
\usepackage{amsmath}
\usepackage{graphicx}
\usepackage{multicol}
\usepackage{amsmath}
\usepackage{tcolorbox}
\usepackage{subcaption}
\usepackage[spanish]{babel}
\usepackage{lipsum}
%%%{{{ Comments and the like
\usepackage[textwidth=4cm]{todonotes}
\usepackage{soul}
\usepackage{xcolor}
\newcounter{todocounter}
\renewcommand{\comment}[2]{\stepcounter{todocounter}
  {\color{green!50!blue}{(#1$^{{\color{black}\textbf{\thetodocounter}}}$)}}
  \todo[color=green,noline,size=\tiny]{\textbf{\thetodocounter:} #2

  }}
\newcommand{\quitaesto}[1]{{\color{red}(\st{#1})}}

\newcommand{\cambio}[2]{{\color{cyan}{{#2}}}{\color{red}{(\st{#1})}}}

\newcommand{\agregaesto}[1]{{\color{cyan}{{#1}}}}

\newcommand{\notaparaelautor}[1]{{\color{brown}{\textbf{#1}}}}

\newcommand{\errorortografico}[1]{{\fcolorbox{gray}{magenta}{\textcolor{yellow}{\bf #1}}}}
    
%%%}}}
\begin{document}

\section{Introducción}
El transporte de cargas terrestre, es una de las formas más esenciales de distribución. Contribuyendo no solo con el abastecimiento de recursos necesarios para el uso diario, sino que además es una excelente forma de generar puestos de trabajo, lo cual ayuda a impulsar la economía de un país. Sin embargo, esta forma de transporte se enfrenta a grandes desafíos en la búsqueda de rentabilidad y eficiencia, dado que el coste promedio de transporte terrestre se estima en \$2.01 dólares por kilómetro recorrido \cite{ref1}.


Debido a estos constantes desafíos, existe una rama de la matemática, dedicada al estudio de problemas de optimización, y específicamente este tipo de problemas se llaman, Problema de Enrutamiento de Vehículos (VRP por sus siglas en inglés). El problema de manera general consiste en: una flota de vehículos, un almacén central y un grupo de clientes a los cuales se les desea entregar los productos del almacén. Lo que se desea es organizar la flota, para que les entreguen a todos los clientes recorriendo la menor distancia posible. 

El problema inicial fue introducido por Dantzig y Ramser en 1959 \cite{ref18}, bajo el nombre de ``El Problema de Despacho de Camiones''. Consistía en la modelación de unos camiones idénticos, los cuales, tenían que servir las demandas de combustible de un número de gasolineras partiendo desde un mismo depósito.

Otras variantes al VRP que han sido estudiadas son: problema de enrutamiento de vehículos con restricciones de capacidad (CVRP), con ventanas de tiempo (VRPTW), con entregas fraccionadas (VPRSD), multi-depósito (MDVRP), aunque hay variantes que modelan mejor el comportamiento de la vida real, como lo son el VRP estocástico \cite{ref2}.


El VRP es un problema NP-Duro, por lo tanto, para encontrar soluciones a problemas de mediana dimensión requiere de un alto costo computacional, por no mencionar aquellos que poseen grandes dimensiones, que en la practica son los que necesitan ser resueltos. Dado que explorar exhaustivamente estas soluciones es altamente costoso, las mejores aproximaciones que se han hecho, han sido a través de metaheurísticas \cite{ref4} y búsqueda aleatoria en los espacios de estas soluciones.

Una de las técnicas que aprovechan estrategias basadas en la exploración inteligente del espacio de soluciones, tales como la Búsqueda en Vecindad Variable (VNS)\cite{ref17}. La VNS es una metaheurística basada en la idea de cambiar dinámicamente la estructura de vecindades durante la búsqueda para escapar de óptimos locales. Comienza con una solución inicial y alterna entre diferentes vecindades, explorándolas con intensificación y diversificación. La fuerza de este enfoque radica en su capacidad para explorar de manera sistemática regiones más amplias del espacio de soluciones, mientras mantiene un equilibrio entre calidad y eficiencia computacional.

Los algoritmos exactos para resolver el problema de enrutamiento de vehículos (VRP) se enfocan en encontrar soluciones óptimas garantizadas, explorando exhaustivamente el espacio de soluciones. A pesar de su elevado costo computacional, permiten resolver instancias pequeñas y evaluar la calidad de las soluciones obtenidas mediante heurísticas y metaheurísticas. En su mayoría estos algoritmos exactos constituyen utilizan técnicas como: Programación Entera y Lineal, Método de Ramificación y Acotamiento (Branch-and-Bound), Ramificación y Corte (Branch-and-Cut)\cite{ref7}.

En los últimos años la Facultad de Matemática y Computación de la Universidad de La Habana se ha dedicado a estudiar estos problemas \cite{ref5,ref6,ref4,ref3}. Una de las investigaciones que se llevaron a cabo fue, la de encontrar una manera de dividir las vecindades de cada solución en regiones, para acortar el espacio de búsqueda y clasificar las regiones donde podrían estar las mejores soluciones \cite{ref6}. 


El objetivo general de este trabajo es desarrollar y analizar diversos modelos de optimización continua diseñados específicamente para explorar vecindades en el espacio de soluciones del problema de enrutamiento de vehículos (VRP). Estos modelos buscan identificar, de manera más eficiente que los algoritmos tradicionales, las regiones del espacio donde es más probable encontrar las mejores soluciones. Asimismo, se pretende determinar la mejor solución alcanzable dentro del marco de los métodos de optimización continua empleados, evaluando su desempeño y comparándolo con alternativas existentes.\\

Para alcanzar este objetivo general se proponen los siguientes objetivos específicos:
\begin{itemize}
\item
Estudiar los criterios de vecindad del VRP.
\item
Generar de manera eficiente todos los datos a usar en los modelos de optimización.
\item
Crear los modelos de optimización continua.
\item
Comparar la velocidad de estos modelos frente a la búsqueda exhaustiva.
\end{itemize}

Este trabajo se encuentra estructurado de la siguiente forma: en el capitulo 1 se presentan los elementos básicos del trabajo: el problema de enrutamiento de vehiculos, los criterios de vecindad, las regiones de una vecindad, la metaheuristica de Búsqueda en Vecindad Variable y una breve introduciion a la biblioteca de software GLPK y al lenguaje de programacion \textit{Lisp}. En el capitulo 2 se presenta una manera de generar sus datos de manera eficiente. El capitulo 3 se presentan los modelos de optimizacion continua asi como los criterios con los que trabaja. en el capitulo 4 se presentan los resultados de utilizar dichos modelos; finalmente se presentan las conclusiones, recomendaciones, trabajos futuros y la bibliografia consultada.
%===========================Referencias===================================
\newpage
\bibliographystyle{plain}
\bibliography{bibliography} 
\end{document}
