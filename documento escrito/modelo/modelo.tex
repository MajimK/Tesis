\documentclass{article}
\usepackage{amsmath}
\usepackage{graphicx}
\usepackage{multicol}
\usepackage{amsmath}
\usepackage{tcolorbox}
\usepackage{subcaption}
\usepackage[spanish]{babel}
\usepackage{lipsum}

\begin{document}

\section {}
En este capítulo se presentan varios modelos de optimización lineal continua para la exploración y obtención de la mejor solución en vecindades especificas del Problema de Enrutamiento de Vehículos. Aunque existen diferentes estrategias para realizar esta exploración como la búsqueda aleatoria, o búsqueda hasta la primera mejora, ninguna de las anteriores garantiza la obtención del óptimo de la vecindad, salvo la exploración exhaustiva. Sin embargo, los modelos de optimización se utilizan para evitar las operaciones de destrucción y reconstrucción de la solución que realizan este tipo de exploraciones, permitiendo encontrar el óptimo en menos tiempo.

Además, estos modelos son utilizados debido a su rapidez computacional en la resolución de problemas complejos gracias a los avances en algoritmos y herramientas de optimización \cite{ref24}. Esto permite encontrar soluciones de alta calidad, incluso en instancias del problema con múltiples restricciones y variables. 


En la sección 1 se presentan algunas características del VRP, que permiten comprender mejor las variables, parámetros y el modelo mismo. En la sección  2 se presentan las variables de cada modelo. En la sección 3 se presentan los parámetros esenciales de cada modelo y en la sección 4 se define el modelo de optimización general.

\section {Características del VRP}

Una de las características del Problema de Enrutamiento de Vehículos es su representación como un grafo dirigido y ponderado.  Sea $(G=V,E)$ el grafo dirigido y ponderado que representa el sistema de clientes y conexiones entre ellos donde cada vértice $v \in V$ representa a un cliente del sistema, mientras que el vértice especial $v_0$ corresponde al depósito central del sistema de enrutamiento. Este nodo actúa como punto de inicio y fin de cada ruta de servicio. Los arcos se expresan mediante pares ordenados de nodos $j = \langle j_s, j_e \rangle$, donde $j_s$ es el nodo de inicio y $j_e$ el nodo final del arco $j$. Cada arco posee una ponderación que representa la distancia entre el nodo inicio y el nodo final.

Para ilustrar esta conversión, mostramos una posible solución al CVRP con dos rutas y 5 clientes (ver Figura~\ref{fig:solucion-cvrp}), y su representación como un grafo (Figura~\ref{fig:grafo-cvrp}).

\begin{figure}[h!]
$R1: 1,2,3$\\
$R2: 4,5$
    \caption{Solución para un CVRP con 5 clientes y 2 vehículos.}
    \label{fig:solucion-cvrp}
\end{figure}

\begin{figure}[h!]
$0 \overset{0}{\rightarrow}1 \overset{1}{\rightarrow}2 \overset{2}{\rightarrow}3 \overset{3}{\rightarrow}0$\\
$0 \overset{0}{\rightarrow}4 \overset{1}{\rightarrow}5 \overset{2}{\rightarrow}0$

    \caption{Representación como grafo.}
    \label{fig:grafo-cvrp}
\end{figure}

En el caso de esta representación grafica, los números encima de cada arco no es la ponderación del arco, si no la posición que ocupa cada arco en su ruta comenzando por 0, de esta manera cada arco corresponde con el número de cliente de su ruta.

La otra característica que es necesario comprender son las \textbf{subrutas} de un VRP. Las subrutas representan fragmentos específicos dentro de una ruta que pueden ser modificados, intercambiados o reinsertados para mejorar la solución general del problema. Formalmente una subruta se define como una secuencia continua de clientes dentro de una misma ruta. Estas subrutas están delimitadas por dos arcos, $j_1$ y $j_2$, lo que significa que la subruta comienza en el cliente siguiente a $j_1$ y termina en el cliente correspondiente a $j_2$. Por ejemplo, en el grafo representado en la figura \ref{fig:grafo-cvrp}, es posible identificar subrutas como las que se muestran en la figura \ref{fig:subrutas-cvrp}

\begin{figure}[h!]
$1 \rightarrow2$\\
$4$\\
$1 \rightarrow2 \rightarrow3 $
    \caption{Ejemplos de subrutas del grafo de la Figura~\ref{fig:grafo-cvrp}.}
    \label{fig:subrutas-cvrp}
\end{figure}


\section {Variables de los modelos}
El modelo general que se presenta considera tres variables continuas cuyos valores se encuentran entre 0 y 1, pero con diferente estructura puesto que, cada una de estas variables esta diseñada para un criterio especifico de vecindad, permitiendo modelar y explorar las diferentes configuraciones posibles dentro del grafo del problema. Los tres criterios que se tomaron fueron: \textit{Reinserción de subruta, intercambio de subrutas e intercambio de clientes.}
\begin{itemize}
\item \textbf{Reinserción de subruta}:
Para la exploración de esta vecindad se utilizan variables denotadas como $X_{r_1 j_1 j_2 r_2 i}$. Estas variables indican si una subruta que comienza en el cliente $j_1 +1$ y termina en el cliente $j_2$ de la ruta $r_1$ es seleccionada y reubicada en la ruta $r_2$ entre los clientes $i$ e $i+1$. 

Es importante destacar que, cuando la ruta de origen $r_1$ coincide con la ruta de destino $r_2$, es decir, cuando la subruta se reubica dentro de la misma ruta, deben cumplirse ciertas restricciones para asegurar la validez de la operación. En este caso la posicion de inserción $i$ debe ser tal que $i < j_1$ o $i>j_2$. Esto evita insertar la subruta en una posición que se encuentra dentro de la misma subruta.

 
\item \textbf {Intercambio de subrutas}:
Estas variables denotadas como $Y_{r_1j_1j_2r_2i_1i_2}$, comparten en estructura similitud con las anteriores. Indican si la subruta que comienza en el cliente $j_1+1$ y termina en el cliente $j_2$ de la ruta $r_1$ debe intercambiarse por la subruta de la ruta $r_2$ que comienza y termina en los clientes $i_1+1$ e $i_2$ respectivamente.

Es importante destacar que, en el caso del intercambio de subrutas, deben cumplirse restricciones adicionales para garantizar la validez de la operación y preservar la estructura del grafo. Cuando la ruta de origen $r_1 = r_2$, es decir, cuando ambas subrutas pertenecen a la misma ruta, las posiciones de las subrutas deben cumplir que $i_1,i_2 \leq j_1$ o $i_1,i_2 \geq j_2$. Esto asegura que las subrutas no interfieran entre sí durante el intercambio. En caso de que $i_1=j_1$ y $i_2 = j_2$, entonces se selecciona la misma ruta y se ubica en la posición en la que estaba.

\item \textbf {Intercambio de clientes}
Las variables para este criterio se denotan como $C_{r_1jr_2i}$, las cuales representan la operación de intercambiar al cliente $j$ de la ruta $r_1$ con el cliente $i$ de la ruta $r_2$. Al igual que en las variables $Y$, en estas si $r_1 = r_2$ y $j=i$, se selecciona el cliente y se mantiene en la posición actual.
\end{itemize}
Para ilustrar cómo funcionan estas variables, consideremos el grafo mostrado en la figura \ref{fig:variables_example}. Este grafo incluye dos rutas y cinco clientes. A continuación, varios ejemplos de cada variable:

\begin{figure}[h!]
    \vspace{0.5em}
$r_1:  0 \rightarrow 1\rightarrow 4 \rightarrow 5 \rightarrow 7 \rightarrow 9 \rightarrow 10 \rightarrow  0$\\
$r_2:  0\rightarrow 2\rightarrow 3\rightarrow 8 \rightarrow 6 \rightarrow  0$\\
\caption{Ejemplos de variables}
\label{fig:variables_example}
\end{figure}


\begin{itemize}
\item $X_{2,0,1,1,1}$: Representa que la subruta formada por el cliente en la segunda posición de la ruta 2 (\([2]\)) será insertada después del primer cliente de la ruta 1.
\item $X_{1,1,3,2,2}$: Define que la subruta comprendida por el segundo y tercer cliente de la ruta 1 (\([4,5]\)) será reubicada en la ruta 2, justo después del cliente en la segunda posición.  
\item $Y_{1,1,3,2,2,4}$: Describe un intercambio donde la subruta \([4,5]\), formada por el segundo y tercer clientes de la ruta 1, será sustituida por la subruta \([8,6]\) que posee al tercer y cuarto cliente de la ruta 2.  
\item $Y_{1,4,6,1,1,3}$: Indica un intercambio dentro de la misma ruta (ruta 1), donde la subruta formada por los clientes \([9,10]\) (posiciones quinta y sexta) será intercambiada por la subruta \([4,5]\) (posiciones segunda y tercera).  
\item $C_{1,5,2,4}$: Este caso particular denota una operación en la que se intercambia directamente el cliente ubicado en la quinta posición de la ruta 1 con el cliente de la posición 24 en otro conjunto.  
\end{itemize}

Los efectos de estas operaciones se muestran en la figura \ref{fig:transformation}, donde se observa cómo cambian las rutas originales al aplicar los valores de las variables.

\begin{figure}[h!]
    \vspace{0.5em}
\begin{enumerate}
\item
$r_1:  0 \rightarrow 1\rightarrow 2\rightarrow 4 \rightarrow 5 \rightarrow 7 \rightarrow 9 \rightarrow 10 \rightarrow 0$\\
$r_2:  0\rightarrow 3\rightarrow 8 \rightarrow 6 \rightarrow 0$\\
\item
$r_1:  0 \rightarrow 1 \rightarrow 7 \rightarrow 9 \rightarrow 10 \rightarrow 0$\\
$r_2:  0\rightarrow 2\rightarrow 3 \rightarrow 4 \rightarrow 5\rightarrow 8 \rightarrow 6 \rightarrow 0$\\
\item
$r_1:  0 \rightarrow 1\rightarrow  8 \rightarrow 6 \rightarrow 7 \rightarrow 9 \rightarrow 10 \rightarrow  0$\\
$r_2:  0\rightarrow 2\rightarrow 3\rightarrow 4 \rightarrow 5 \rightarrow  0$\\
\item 
$r_1:  0 \rightarrow 1\rightarrow 9 \rightarrow 10 \rightarrow 7 \rightarrow 4 \rightarrow 5 \rightarrow  0$\\
$r_2:  0\rightarrow 2\rightarrow 3\rightarrow 8 \rightarrow 6 \rightarrow  0$\\
\item
$r_1:  0 \rightarrow 1\rightarrow 4 \rightarrow 5 \rightarrow 7 \rightarrow 6 \rightarrow 10 \rightarrow  0$\\
$r_2:  0\rightarrow 2\rightarrow 3\rightarrow 8 \rightarrow 9 \rightarrow  0$\\
\end{enumerate}
\caption{Transformación hecha por la variable: 1-) $X_{2,0,1,1,1}$ 2-) $X_{1,1,3,2,2}$ \\3-) $Y_{1,1,3,2,2,4}$ 4-)$Y_{1,4,6,1,1,3}$  5-) $C_{1,5,2,4}$}
\label{fig:transformation}
\end{figure}










Además de las variables continuas descritas anteriormente, el modelo utiliza una serie de parámetros que permiten evaluar las características de las rutas y los efectos de las manipulaciones de subrutas. Estos parámetros permiten cuantificar los costos, demandas y conexiones dentro del grafo de solución. 

\section{Parámetros}

Los parámetros descritos a continuación son utilizados en cada modelo:

\begin{enumerate}
\item{$capacity$: Representa la capacidad máxima de cada vehículo. La demanda total de cualquier ruta no puede exceder este límite. En la próxima sección se analizara el caso en que se exceda.}
\item{$d_i$: Este parámetro representa la demanda del cliente $i$. La demanda de cada cliente no puede exceder la capacidad de un vehículo, o sea \\ $\forall {i}, d_i < capacity$ }
\item{$c_{rj}$: Indica el peso del arco $j$ en la ruta $r$. Este parámetro refleja la función de ponderación asignada a los arcos del grafo $G$.}


\item{$P_r$: Representa el peso total de la ruta $r$ para una solución dada. Este se calcula como la suma de los pesos de todos los arcos de la ruta:}
  \[
P_r=\sum c_{rj}  
\]

\item{$S_{rj_1j_2}$:Define el peso total de los arcos de una subruta comprendida entre los clientes $j_1+1$ y $j_2$. Se calcula mediante:}

\[
S_{rj_1j_2}=\sum\limits_{i=j_1+1}^{j_2-1} c_{ri} 
\]
\begin{itemize}
\item
Si la subruta contiene solo un cliente, este parámetro es igual a 0.
\item
Si la subruta incluye todos los clientes de una ruta, el valor de \\ $S_{rj_1j_2} = P_r$  menos el costo de entrada y salida del depósito.
\end{itemize}

\item {$D_{rj_1j_2}$: Representa la demanda total de la subruta desde el cliente en la posición $j_1+1$ hasta el cliente en la posición $j_2$ de la ruta $r$. Se define como:}

\[
D_{rj_1j_2}=\sum\limits_{i=j_1+1}^{j_2-1} d_i 
\]
\item{$K_{r_1ir_2j}$:Representa el peso del arco $\langle i+1, j \rangle$, donde $i$ pertenece a la ruta $r_1$ y $j$ pertenece a la ruta $r_2$.}
\item{$L_{r_1ir_2j}$: Representa el peso del arco $\langle i, j+1 \rangle$, con los mismos conjuntos de clientes que $K_{r_1ir_2j}$.}\\

Estos dos parámetros ($K$ y $L$) permiten evaluar las operaciones de inserción y eliminación de subrutas. Conectan los nodos restantes al eliminar una subruta, así como los extremos de la subruta con los clientes correspondientes. En la figura \ref{fig:parameter graph}, se muestra un ejemplo de grafo solución con las aristas ponderadas, mientras que en la figura \ref{fig:K-L examples} se ilustran visualmente estos parámetros.

\begin{figure}[h!] \vspace{0.5em} $r_1: 0 \overset{5}{\rightarrow} 1\overset{8}{\rightarrow} 2\overset{3}{\rightarrow} 3\overset{6}{\rightarrow} 0$\\
$r_2: 0\overset{1}{\rightarrow} 4\overset{2}{\rightarrow} 5\overset{9}{\rightarrow} 0$
\caption{Representación como grafo} 
\label{fig:parameter graph} 
\end{figure}

\begin{figure}[h!] 
\vspace{0.5em} 
\begin{enumerate} 
\item $1 \overset{1}{\rightarrow} 5 $ 
\item $5 \overset{2}{\rightarrow} 3 $ 
\end{enumerate} 
\caption{a-) $L_{1,1,2,1}$; b-) $K_{2,1,1,3}$}
 \label{fig:K-L examples} \end{figure}

\item{$RouteDemand_r$: Define la demanda total de la ruta $r$ en una solución dada. Se obtiene mediante:}
\[
RouteDemand_r = \sum\limits_{j=0}^{n-1} d_{i_e}
\] 

\end{enumerate}

Los parámetros descritos no solo capturan las características individuales de las rutas y arcos, sino que también permiten evaluar el impacto de las operaciones de eliminación e inserción de subrutas. Esto facilita la identificación de configuraciones óptimas de rutas en términos de peso total y demanda. Además, al precomputar estos valores, se mejora la eficiencia de la implementación del modelo en problemas de enrutamiento de vehículos, asegurando que las modificaciones sean valoradas adecuadamente en el contexto del sistema global.

Luego de todas estas definiciones, podemos estructurar el modelo matemático.

\section{El modelo}

El modelo tiene como objetivo minimizar el costo total asociado con las operaciones de modificación de las rutas actuales. Este costo incluye tanto el costo de eliminar subrutas de una ruta original como el costo de insertar estas subrutas en otra ruta diferente. Por lo tanto el modelo se define como:

\begin{equation}
\min \sum (c \cdot x) +penalty
\label{eq:objective_function}
\end{equation}
Sujeto a:
\begin{equation}
\sum \limits_{r_1j_1j_2r_2i} X_{{r_1}{j_1}{j_2}{r_2}{i}} = 1 
\label{eq:constraint}
\end{equation}

En la ecuación \ref{eq:objective_function} $x$ es la variable continua y $c$ su coeficiente, el factor penalty es un factor de penalidad, mas adelante se define. Al igual que las variables, la forma de calcular $c$ varía para cada criterio de vecindad considerado. Cada coeficiente se determina en función del impacto que genera la operación asociada en el modelo, tomando en cuenta las características específicas de la vecindad. Antes de describir el calculo de $c$ es necesario considerar dos componentes que describen los costos asociados a las operaciones de reubicación de subrutas:

\begin{itemize}
\item {$Eliminar$: Esta componente modela el coste asociado con la eliminación de una subruta especifica, definiendo asi el nuevo peso de la ruta $r_1$. Se modela como la eliminación del peso de la subruta (El parámetro S) así como los arcos de los extremos de la misma, para añadir el arco entre los clientes justo fuera de la ruta:}
\[
Eliminar_{r_1j_1j_2r_2i}=\sum P_{r_1}-S_{{r_1}{j_1}{j_2}}-c_{{r_1}{j_1}}-c_{{r_1}{j_2}}+L_{{r_1}{j_1}{r_1}{j_2}}
\]
\item {$Sumar$: Este componente modela el coste asociado con la adición de una subruta en una nueva posición dentro de la ruta $r_2$, otorgándole un nuevo valor al peso de esa ruta. Esta operación separa las clientes $i,i+1$ para insertar la subruta en esta posicion, uniendo el inicio de esta con $i$ y el final con $i+1$:}
 \[
Sumar_{r_1j_1j_2r_2i}=\sum  P_{r_2} - c_{{r_2}{i}} + S_{{r_1}{j_1}{j_2}}+K_{{r_1}{j_1}{r_2}{i}} + L_{{r_1}{j_2}{r_2}{i}}
\] 
\end{itemize}
A continuación, se detalla el calculo de $c$ para cada criterio:
\begin{enumerate}

\item \textbf{Reinserción de subruta}:
Con las dos componentes, Eliminar y Sumar, se evalúan de manera independiente los costos asociados con la extracción y la inserción de la subruta. Al combinarlas estamos listos para calcular el coeficiente $c$, que refleja el costo total de realizar el movimiento en la vecindad. El coeficiente se define como:
\[
c= \sum \limits _{r\neq r_1,r_2} P_r + Eliminar + Sumar
\]

Aquí $\sum \limits _{r\neq r_1,r_2} P_r$, representa el costo total de las rutas no afectadas por las componentes, asegurando que solo se modifiquen los costos asociados a las rutas modificadas ($r_1 y r_2$). 

\item \textbf {Intercambio de subrutas}:
En este criterio, se evalúa el costo de intercambiar dos subrutas entre las rutas $r_1$ y $r_2$. Para modelar esta operación, es necesario calcular los costos asociados con:  
\begin{itemize}
\item \textbf{Eliminar}: La extracción de la subruta de seleccionada de $r_1$ y la seleccionada de $r_2$.  
\item \textbf{Sumar}: La inserción de la subruta de $r_2$ en $r_1$ y viceversa.  
\end{itemize}

El coeficiente $c$ se define como la suma de estos costos:  
\[
c = \sum \limits _{r\neq r_1,r_2} P_r + Eliminar_{r_1} + Eliminar_{r_2} + Sumar_{r_1 \to r_2} + Sumar_{r_2 \to r_1} 
\]

Aquí, las componentes $Eliminar_{r_1}$ y $Eliminar_{r_2}$ modelan la extracción de las subrutas respectivas, mientras que $Sumar_{r_1 \to r_2}$ y $Sumar_{r_2 \to r_1}$ consideran la adición de las subrutas intercambiadas. Este modelo asegura que las rutas restantes ($r \neq r_1, r_2$) no se vean afectadas.  


\item \textbf {Intercambio de clientes:}
En este criterio, el costo está asociado al intercambio de clientes individuales entre dos rutas $r_1$ y $r_2$. Este proceso puede entenderse como un caso particular del intercambio de subrutas, donde la subruta tiene longitud uno. Los costos de eliminar al cliente de su posición original y reubicarlo en una nueva se modelan con las mismas componentes:  

El coeficiente $c$ para esta operación se calcula como:  
\[
c = \sum \limits _{r\neq r_1,r_2} P_r + Eliminar_{r_1} + Eliminar_{r_2} + Sumar_{r_1 \to r_2} + Sumar_{r_2 \to r_1} 
\]

Dado que solo se trabaja con un cliente en este caso, los términos asociados a $Eliminar$ y $Sumar$ están simplificados en comparación con los cálculos de subrutas completas, dado que en este caso el parámetro $S=0$, pero el principio subyacente permanece el mismo.
\end{enumerate}

En la ecuación (1), $penalty$ es un factor de penalidad, que se le aplica a la función objetivo cuando la solución en la que nos encontramos es infactible, es decir no cumple la restricción de capacidad en el caso del CVRP la cual se define como:
\begin{equation}
\sum \limits_{i=1}^{n} d_i < capacity
\end{equation}

Con esto, se han definido los coeficientes $c$ para cada criterio de vecindad considerado, como se describió previamente. Estos coeficientes, detallados en función de las operaciones de modificación en las rutas actuales, representan el impacto directo de dichas operaciones y proporcionan una base sólida para modelar el costo total del sistema. 

Sin embargo, como se mencionó en la ecuación \ref{eq:objective_function}, el modelo también incluye un término de penalidad ($penalty$), cuya función es garantizar que las soluciones infactibles  no sean elegidas como solución óptima. Al penalizar a la función objetivo, se busca simplificar las restricciones del modelo y dirigir la búsqueda hacia soluciones factibles y de alta calidad. Una solución es infactible cuando se viola alguna de las restricciones del problema especifico, en este caso la restricción de capacidad:
\[
\exists r \; \text{;} \; RouteDemand_r > capacity 
\]

El parámetro $penalty$ se define de la siguiente manera:
\[
penalty = 
\begin{cases}
    \alpha \cdot  (D_{rj_1j_2} + RouteDemand - capacity) & \text{Si al insertar la subruta, la solución es infactible} \\
    \alpha \cdot (RouteDemand - D_{rj_1j_2} - capacity) & \text{Si al eliminar la subruta, la solución continua siendo}\\ 
	& \text{infactible}\\
    0 & \text{Si la solución es factible} 
\end{cases}
\]

Aquí, $\alpha$ es un factor de penalización que determina la severidad de la violación de la restricción de capacidad. Este factor puede ajustarse para priorizar la factibilidad sobre el costo total, dependiendo de los objetivos específicos del modelo.

Este término asegura que las soluciones que no respeten las restricciones de capacidad sean penalizadas suficientemente en la función objetivo, favoreciendo así configuraciones factibles sin necesidad de imponer restricciones complejas adicionales.

\newpage
\bibliographystyle{plain}
\bibliography{bibliography} 

\end{document}