\documentclass[12pt]{report}
\usepackage{amsmath}
\usepackage{graphicx}
\usepackage{multicol}
\usepackage{amsmath}
\usepackage{tcolorbox}
\usepackage{subcaption}
\usepackage[spanish]{babel}
\usepackage{lipsum}
\usepackage{xcolor}

\usepackage[lmargin=2cm,rmargin=5cm]{geometry}

%%%{{{ Comments and the like
\usepackage[textwidth=4cm]{todonotes}
\usepackage{soul}
\usepackage{xcolor}
\newcounter{todocounter}
\renewcommand{\comment}[2]{\stepcounter{todocounter}
  {\color{green!50!blue}{(#1$^{{\color{black}\textbf{\thetodocounter}}}$)}}
  \todo[color=green,noline,size=\tiny]{\textbf{\thetodocounter:} #2

  }}
\newcommand{\quitaesto}[1]{{\color{red}(\st{#1})}}

\newcommand{\cambio}[2]{{\color{cyan}{{#2}}}{\color{red}{(\st{#1})}}}

\newcommand{\agregaesto}[1]{{\color{cyan}{{#1}}}}

\newcommand{\notaparaelautor}[1]{{\color{brown}{\textbf{#1}}}}

\newcommand{\errorortografico}[1]{{\fcolorbox{gray}{magenta}{\textcolor{yellow}{\bf #1}}}}
    
%%%}}}


\begin{document}

	Como trabajo futuro, se propone explorar estrategias para construir los datos de manera incremental, generándolos a medida que avanza el algoritmo en lugar de calcular toda la vecindad desde el inicio. Esto permitiría reducir la cantidad de cálculos innecesarios.

	También se recomienda desarrollar técnicas eficientes para la generación de datos en otros tipos de vecindades, extendiendo el enfoque utilizado en este trabajo a vecindades más complejas.

	Otra línea investigación sería el diseño de métodos para estimar los parámetros de algunos vecinos sin necesidad de calcularlos de manera explícita. El uso de aproximaciones basadas en técnicas estadísticas o aprendizaje automático podría permitir reducir el costo computacional de la evaluación de soluciones vecinas.

	Los próximos trabajos deben estar dirigidos en estas direcciones.
\end{document}