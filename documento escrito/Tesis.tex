\documentclass{report}
\usepackage{amsmath}
\usepackage{graphicx}
\usepackage{multicol}
\usepackage{amsmath}
\usepackage{tcolorbox}
\usepackage{subcaption}
\usepackage[spanish]{babel}
\usepackage{lipsum}

\title {Un acercamiento al VRP y una propuesta de solución}
\author {Kevin Majim Ortega Alvarez}

\begin{document}
\maketitle
\section {Introducción}

El transporte de cargas terrestre, es una de las formas más esenciales de distribución. Contribuyendo no solo con el abastecimiento de recursos necesarios para el uso diario, sino que además es una excelente forma de generar puestos de trabajo, lo cual ayuda a impulsar la economía de un país. Sin embargo, esta forma de transporte se enfrenta a grandes desafíos en la búsqueda de rentabilidad y eficiencia, dado que el coste promedio de transporte terrestre se estima en \$2.01 dólares por kilómetro recorrido \cite{ref1}.


Debido a estos constantes desafíos, existe una rama de la matemática, dedicada al estudio de problemas de optimización, y específicamente este tipo de problemas se llaman, Problema de Enrutamiento de Vehículos (VRP por sus siglas en inglés). El problema de manera general consiste en: una flota de vehículos, un almacén central y un grupo de clientes a los cuales se les desea entregar los productos del almacén. Lo que se desea es organizar la flota, para que les entreguen a todos los clientes recorriendo la menor distancia posible. 


Otras variantes al VRP que han sido estudiadas son: problema de enrutamiento de vehículos con restricciones de capacidad (CVRP), con ventanas de tiempo (VRPTW), con entregas fraccionadas (VPRSD), multi-depósito (MDVRP), aunque hay variantes que modelan mejor el comportamiento de la vida real, como lo son el VRP estocástico \cite{ref2}.


El VRP es un problema NP-Duro \cite{ref3}, por lo tanto, para encontrar soluciones a problemas de mediana dimensión requiere de un alto costo computacional, por no mencionar aquellos que poseen grandes dimensiones, que en la practica son los que necesitan ser resueltos. Dado que explorar exhaustivamente estas soluciones es altamente costoso, las mejores aproximaciones que se han hecho, han sido a través de metaheurísticas \cite{ref} y búsqueda aleatoria en los espacios de estas soluciones.


En los últimos años la Facultad de Matemática y Computación de la Universidad de La Habana se ha dedicado a estudiar estos problemas \cite{ref5}\cite{ref6}. Una de las investigaciones que se llevaron a cabo fue, la de encontrar una manera de dividir las vecindades de cada solución en regiones, para acortar el espacio de búsqueda y clasificar las regiones donde podrían estar las mejores soluciones \cite{ref7}. 


El objetivo general de este trabajo es crear un modelo de optimización que encuentre, las regiones donde se encuentran las mejores soluciones, para una vez detectada, realizar una búsqueda por ella.\\
A continuación presentamos el modelo
\section {Definir ciertas cosas}

Lo que definiremos aquí son algunas generalizaciones que vamos a utilizar en ambos modelos:

\begin{enumerate}
\item {$ r \rightarrow $ Rutas}

\item {$i,j \rightarrow$ Arcos donde $j=<j_s,j_e>$ y $j_s$ es el nodo inicial del arco $j$ y $j_f$ el final}

\item {$G=\{V,E\} \rightarrow$ Grafo dirigido y ponderado donde para todo $v \in V$ se cumple que $v$ es un cliente de nuestro sistema y si $<v_1,v_2> \in E$
hay un arco de $v_1$ a $v_2$. G es conexo y $v_0$ es el único punto de articulación  por lo que todo nodo pertenece a un ciclo que empieza y termina en $v_0$. Si eliminamos $v_0$ entonces se forman varias Componentes Conexas ($CC$ para los amigos), cada $CC$ representa una ruta distinta en nuestro sistema. Luego podemos enumerarlas}

\item {Se puede establecer una especie de orden entre los arcos definiendo que $i<j$ si $i$ aparece primero en una ruta que $j$}

\end{enumerate}

Hasta aquí lo que tienen en común ambos modelos. 
\section {Modelo (X)}

\subsection{Variables del sistema}



\begin{enumerate}
\item
$
 X_{r_1 j_1 j_2 r_2 i}= 
\begin{cases}  
    1 & \text{si eliminamos $j_1,j_2$ de $r_1$ e insertamos en $i$ de $r_2$ ($j_1< j_2$) }\\
    0 & \text{si no } 
\end{cases}
$
\item{$c_{rj}\rightarrow$ Peso de $j$ en $r$}

\item{$S_{rij_1}\rightarrow$ Suma de los pesos de los arcos a partir de $i+1$ hasta $j-1$}\\
($S_{rij_1}=\sum\limits_{j_2=i+1}^{j_1-1} c_{rj_2}$ )

\item{$K_{r_1ir_2j}\rightarrow$ Peso del arco $<j_s,i_e>$ donde $i$ se encuentra en la ruta $r_1$ y $j$ en la ruta $r_2$}

\item{$L_{r_1ir_2j}\rightarrow$ Peso del arco $<i_s,j_e>$ donde $i$ se encuentra en la ruta $r_1$ y $j$ en la ruta $r_2$}

\item{$P_r\rightarrow$ Peso total de $r$ para esta solución ($P_r=\sum\limits_{j=1}^{n} c_{rj}$ ) }

\item{$Eliminar=\sum\limits_{r_1=1}^{n_{r}} \sum\limits_{j_1=1}^{n_{j_1}}\sum\limits_{j_2=1}^{n_{j_2}}\sum\limits_{r_2=1}^{n_{r}}\sum\limits_{i=1}^{n_i} X_{{r_1}{j_1}{j_2}{r_2}{i}}*(P_{r_1}-S_{{r_1}{j_1}{j_2}}-c_{{r_1}{j_1}}-c_{{r_1}{j_2}}+L_{{r_1}{j_1}{r_1}{j_2}})$}

\item{$Sumar=\sum\limits_{r_1=1}^{n_{r}} \sum\limits_{j_1=1}^{n_{j_1}}\sum\limits_{j_2=1}^{n_{j_2}}\sum\limits_{r_2=1}^{n_{r}}\sum\limits_{i=1}^{n_i} X_{{r_1}{j_1}{j_2}{r_2}{i}}*(P_{r_2} - c_{{r_2}{i}} + S_{{r_1}{j_1}{j_2}}+K_{{r_1}{j_1}{r_2}{i}} + L_{{r_1}{j_2}{r_2}{i}}) $}
\end{enumerate}
\subsection{El modelo}
	Función objetivo y restricciones propias del VRP.\\
Las que añadimos nosotros:
\begin{enumerate}
\item{$\sum\limits_{i=j_1}^{j_2} X_{rj_1j_2ri}=0$  $ \forall r$, $\forall j_1<j_2 $}
\item{$\sum P_r> Eliminar + Sumar $}
\item{$ \sum\limits_{r_1=1}^{n_{r}} \sum\limits_{j_1=1}^{n_{j_1}}\sum\limits_{j_2=1}^{n_{j_2}}\sum\limits_{r_2=1}^{n_{r}}\sum\limits_{i=1}^{n_i}X_{{r_1}{j_1}{j_2}{r_2}{i}} = 1 $}
\end{enumerate}
%===========================Referencias===================================
\bibliographystyle{plain}
\bibliography{bibliography} 
\end{document}
