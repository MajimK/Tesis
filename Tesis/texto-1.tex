\documentclass{article}
\usepackage{amsmath}


\begin{document}
\section {Definir ciertas cosas}

Lo que definiremos aqui son algunas generalizaciones que vamos a utilizar en ambos modelos:

\begin{enumerate}
\item {$ r \rightarrow $ Rutas}

\item {$i,j \rightarrow$ Arcos donde $j=<j_s,j_e>$ y $j_s$ es el nodo inicial del arco $j$ y $j_f$ el final}

\item {$G=\{V,E\} \rightarrow$ Grafo dirigido y ponderado donde para todo $v \in V$ se cumple que $v$ es un cliente de nuestro sistema y si $<v_1,v_2> \in E$
hay un arco de $v_1$ a $v_2$. G es conexo y $v_0$ es el unico punto de acumulacion por lo que todo nodo pertenece a un ciclo que empieza y termina en $v_0$. Si eliminamos $v_0$ entonces se forman varias Componentes Conexas ($CC$ para los amigos), cada $CC$ representa una ruta distinta en nuestro sistema. Luego podemos enumerarlas.}

\item {Se puede establecer una especie de orden entre los arcos definiendo que $i<j$ si $i$ aparece primero en una ruta que $j$}

\end{enumerate}

Hasta aqui lo que tienen en comun ambos modelos.
\section {Modelo (X)}

\subsection{Variables del sistema}



\begin{enumerate}
\item
$
 X_{r_1 j_1 j_2 r_2 i}= 
\begin{cases}  
    1 & \text{si eliminamos $j_1,j_2$ de $r_1$ e insertamos en $i$ de $r_2$ ($j_1< j_2$) }\\
    0 & \text{si no } 
\end{cases}
$
\item{$c_{rj}\rightarrow$ Peso de $j$ en $r$}

\item{$S_{rij_1}\rightarrow$ Suma de los pesos de los arcos a partir de $i+1$ hasta $j-1$}\\
($S_{rij_1}=\sum\limits_{j_2=i+1}^{j_1-1} c_{rj_2}$ )

\item{$K_{r_1ir_2j}\rightarrow$ Peso del arco $<j_s,i_e>$ donde $i$ se encuentra en la ruta $r_1$ y $j$ en la ruta $r_2$}

\item{$L_{r_1ir_2j}\rightarrow$ Peso del arco $<i_s,j_e>$ donde $i$ se encuentra en la ruta $r_1$ y $j$ en la ruta $r_2$}

\item{$P_r\rightarrow$ Peso total de $r$ para esta solucion ($P_r=\sum\limits_{j=1}^{n} c_{rj}$ ) }

\item{$Eliminar=\sum\limits_{r_1=1}^{n_{r}} \sum\limits_{j_1=1}^{n_{j_1}}\sum\limits_{j_2=1}^{n_{j_2}}\sum\limits_{r_2=1}^{n_{r}}\sum\limits_{i=1}^{n_i} X_{{r_1}{j_1}{j_2}{r_2}{i}}*(P_{r_1}-S_{{r_1}{j_1}{j_2}}-c_{{r_1}{j_1}}-c_{{r_1}{j_2}}+L_{{r_1}{j_1}{r_1}{j_2}})$}

\item{$Sumar=\sum\limits_{r_1=1}^{n_{r}} \sum\limits_{j_1=1}^{n_{j_1}}\sum\limits_{j_2=1}^{n_{j_2}}\sum\limits_{r_2=1}^{n_{r}}\sum\limits_{i=1}^{n_i} X_{{r_1}{j_1}{j_2}{r_2}{i}}*(P_{r_2} - c_{{r_2}{i}} + S_{{r_1}{j_1}{j_2}}+K_{{r_1}{j_1}{r_2}{i}} + L_{{r_1}{j_2}{r_2}{i}}) $}
\end{enumerate}
\subsection{El modelo}
	Funcion objetivo y restricciones propias del VRP.\\
Las que añadimos nosotros:
\begin{enumerate}
\item{$\sum\limits_{i=j_1}^{j_2} X_{rj_1j_2ri}=0$  $ \forall r$, $\forall j_1<j_2 $}
\item{$\sum P_r> Eliminar + Sumar $}
\item{$ \sum\limits_{r_1=1}^{n_{r}} \sum\limits_{j_1=1}^{n_{j_1}}\sum\limits_{j_2=1}^{n_{j_2}}\sum\limits_{r_2=1}^{n_{r}}\sum\limits_{i=1}^{n_i}X_{{r_1}{j_1}{j_2}{r_2}{i}} = 1 $}
\end{enumerate}
\end{document}
